%% Eliminar las opciones del paquete Resume para cambiar las propiedades de tu currículum, las opciones son:
%% TODO
%% iconos: Para usar iconos de Font Awesome para cosas como: enlaces, repositorios de GitHub, iconos de contacto (correo electrónico, teléfono, LinkedIn, etc.)
%% colores: Hay solo dos colores, pero puedes eliminar la segunda opción utilizada principalmente para los encabezados para eliminar esta opción

\documentclass[9pt,a4paper,icons,colors]{Resume}

\geometry{left=1cm,right=1cm,top=0.8cm,bottom=0.8cm,columnsep=0.6cm}

%% Coloca tu fuente favorita aquí
\setmainfont{Lato}

%% Define colores personalizados, por ejemplo:

%% Para encabezados y separadores principales
\definecolor{VividPurple}{HTML}{3E0097}

%% Para subencabezados y enlaces
\definecolor{SlateGrey}{HTML}{1E1E1E}

%% Texto principal
\definecolor{LightGrey}{HTML}{555555}

%% Balas de tecnología
\definecolor{Black}{HTML}{60b41f}

%% Aplicar colores

\colorlet{heading}{VividPurple}
\colorlet{headingrule}{VividPurple}
\colorlet{accent}{VividPurple}
\colorlet{emphasis}{SlateGrey}
\colorlet{body}{LightGrey}
\colorlet{technologies}{Black}

\begin{document}
\name{Alberto Silva Cazares}
\tagline{Ingeniero en sistemas computacionales / Full Stack developer}

%% Los campos disponibles son
%% \email{}
%% \mailaddress{}
%% \phone{}
%% \homepage{}
%% \twitter{}
%% \linkedin{}
%% \github{}
%% \orcid{}
%% \location{}
%% \medium{}

%% Puedes usar cualquier otro icono con el comando \printinfo{\fa<NombreDelIcono>}{Texto}
%% Todos los iconos provienen de Font Awesome, consulta http://mirrors.ibiblio.org/CTAN/fonts/fontawesome/doc/fontawesome.pdf

\personalinfo{
  \email{silvacazaresluis@outlook.com}
  \phone{+52 5540969864}
  \location{CDMX, México}
  \homepage{alberto-sc.vercel.app}
  \linkedin{alberto-silva-358955175}
  \github{alberto-sc} 
  \english{MCER B2}
}

\makecvHeader

\columnratio{0.6}
%% Comienza el contexto de columnas, la primera columna es la primera que aparece.
\begin{paracol}{2}

\cvsection{Acerca de mí}
Ingeniero de software con sólidos conocimientos en desarrollo de software, apasionado por resolver problemas complejos que requieren optimización o desarrollo de algoritmos complejos. Dispuesto a asumir nuevos desafíos y colaborar en proyectos retadores con el objetivo de seguir creciendo como profesional y contribuir al mundo de la tecnología con soluciones eficientes.

\cvsection{Experiencia relevante}
%% Para crear un proyecto, usa el siguiente comando:
%% \cvproject{Nombre}{Fecha}{EnlaceDelProyecto, sitio web, repositorio o también puede estar vacío}
%% {Puntos separados por | }
%% {Habilidades/Tecnologías utilizadas}
\cvproject
{Dochain (Backend)}{Ethereum Developer Program 2022}
{\website{https://dochain.vercel.app/}{dochain.vercel.app} \githublink{https://github.com/Dochain1}}
{Desarrollo de una DAPP que permite almacenar archivos privados y compartibles en blockchain }
{Integración y pruebas de contratos inteligentes para el almacenamiento seguro de documentos.|
Desarrollo de una arquitectura pensada en la privacidad y confidencialidad de los datos. |
Creación de una API REST} 
{Node.js, Truffle suite, Ethereum, SQL, IPFS, GraphQL}

\divider

\cvproject
{Safemessenger}{2021/22}
{\website{https://safemessenger.xyz}{Safemessenger.xyz} (servicio caído debido a costos)}
{Proyecto de titulación, creación de una inteligencia artificial para clasificar texto con el objetivo de prevenir el bullying en redes sociales}
{Diseño, creación y entrenamiento de redes neuronales convolucionales |
Uso de web sockets para mensajes en tiempo real |
Creación de un servicio de mensajería utilizando React.js}
{Node.js, React.js, Socket.io, Python, Tensorflow, Flask, Nginx}
\divider

\cvproject
{Freelance}{2021}
{}
{Desarrollo de una aplicación web interna para administrar las finanzas de los empleados}
{Creación de un panel CRUD para administrar la información de los empleados | 
Implementación, administración y despliegue en un entorno privado. |
Creación de informes en formato PDF listos para imprimir}
{HTML, CSS, Python, Django, SQL, Linux}
\divider

\cvproject
{Implementación web del juego de la vida de Conway}{2020}
{\githublink{https://github.com/alberto-sc/Game-Of-life}}
{Implementación del famoso juego de la vida de Conway}
{Una implementación parcial del algoritmo hashlife para obtener optimización de espacio con espacio infinito |
Desarrollo de la estructura de datos QuadTree para mejorar el rendimiento |
Uso de D3.js para la visualización de conjuntos de datos grandes}
{Javascript, HTML canvas, D3.js}
\divider

\cvproject
{Referencia competitiva}{desde 2020}
{\githublink{https://github.com/Alberto-SC/Reference}}
{Una colección de varios algoritmos útiles para programación competitiva, así como soluciones a problemas/concursos de programación}
{Más de 2000+ soluciones a desafíos de programación |
Más de 20+ concursos ICPC incluidas finales regionales de diferentes países | 
Colección de algoritmos y estructuras de datos básicos y avanzados de la mayoría de áreas de algoritmos, incluyendo grafos, matemáticas, cadenas, optimización, etc., en su mayoría implementados por mí}
{C++, Python, Go, Javascript, Rust, SQL, LaTeX}

\switchcolumn

\cvsection{Educación}

\cvevent{Ingeniería en Sistemas Computacionales}{Escuela Superior de Cómputo}{2017 - 2022- 2023 en trámite}{}
\cvevent{Ethereum Developer Program}{Platzi - Fundación Ethereum}{mayo de 2022 - octubre de 2022}{}
\cvevent{Educación en línea}{Platzi, FreeCodeCamp}{2020 - en curso}{}

\cvsection{Habilidades técnicas}

%% cvskills toma una lista de elementos separados por comas y coloca cada uno en una etiqueta redondeada. 
\cvskills{Desarrollo Full Stack}{Javascript, HTML y CSS, React.js, Node.js
, Next.js, GraphQL, SQL (PostgreSQL), MongoDB, Prisma (ORM)}
\cvskills{Blockchain}{Solidity, Hardhat,
Truffle suite, Web3.js / Ether.js}
\cvskills{Relacionado con DevOps}{Administración de servidores Linux, Docker, Nginx, Azure DevOps, Git, GitHub Actions}
\cvskills{Cloud}{AWS, Azure (Certificado AZ-900)}

\cvsection{Lenguajes de programación}
\cvlanguaje{Javascript}{3+ años}{Principalmente para desarrollo full stack con React.js y Node.js}
\cvlanguaje{Python}{4+ años}{Principalmente para programación competitiva, además de: web frameworks, ML frameworks, scripting }%% Soporta valores X.5.
\cvlanguaje{C++}{6+ años}{Principalmente para programación competitiva}
\cvlanguaje{LaTeX}{3+ años}{Creación de documentos y material educativo} 
\cvlanguaje{Bash/Lua}{1+ años}{Scripting básico-intermedio para herramientas personales}
\cvlanguaje{Golang}{medio año}{Servicios backend}

\cvsection{Premios y actividades}
\cvachievement{\faTrophy}{Participante en ICPC}{ 
  Participante en el gran premio de México 2022/23 y finalista regional ubicándome en el top 6° de más de 300 equipos como mejor resultado individual y 9° como mejor posición nacional global (circuito de 4 concursos).
  También participé en 2021, 2020, 2019 y 2018; para otros años, se puede consultar en \urlNewWindow{https://icpc.global/ICPCID/44QVI83TR89J}.
}
\divider

\cvachievement{\faCode}{Programador competitivo}{ 
  Como parte de mi entrenamiento para el concurso ICPC, y como pasatiempo, participo habitualmente en la mayoría de las plataformas competitivas más famosas, como Codeforces, CodeChef, LeetCode, donde formo parte del 1\% superior en todo el mundo en algunas de ellas, como LeetCode, donde actualmente ocupo la posición 880/400k+. Compito regularmente con más de 100k+ usuarios activos. Puedes encontrar mis otras clasificaciones en \url{https://clist.by/coder/BetoSCL/}.
}
\divider

\cvachievement{\faChalkboardTeacher}{Profesor}{ 
  Como contribución a la comunidad, fui profesor en temas relacionados con programación competitiva, como estructuras de datos avanzadas, algoritmos de cadenas y búsqueda binaria, buscando ofrecer una preparación de alta calidad para la competición ICPC para obtener un lugar en las finales regionales. También contribuyo y he ofrecido mentoría en estructuras de datos y algoritmos en plataformas de aprendizaje, como LeetCode, se pueden encontrar algunas de mis contribuciones en mi perfil de leetcode \url{https://leetcode.com/BetoSCL/}{perfil}.
}

\end{paracol}
\end{document}
